%%
%% Copyright (c) 2001, 2009, 2010 The American Physical Society.
%%
%% See the REVTeX 4 README file for restrictions and more information.
%%
%To achieve the polarity reversal, several techniques have already been employed:
%magnetic field bursts6, oscillating perpendicular magnetic fields7 and %in-plane rotating magnetic field8, among others. We decided to employ a %rotating magnetic field due to the low intensity of the applied field and %the frequency selectivity available in the process, i.e., the vortex core %will only reverse for a well-defined range of field frequencies12. It was %also noted that the gyrotropic frequency decreased for increasing values of %K.


% This is a template for producing manuscripts for use with REVTEX 4.0
% Copy this file to another name and then work on that file.
% That way, you always have this original template file to use.
%
% Group addresses by affiliation; use superscriptaddress for long
% author lists, or if there are many overlapping affiliations.
% For Phys. Rev. appearance, change preprint to twocolumn.
% Choose pra, prb, prc, prd, pre, prl, prstab, prstper, or rmp for journal
% Add 'draft' option to mark overfull boxes with black boxes
% Add 'showpacs' option to make PACS codes appear
% Add 'showkeys' option to make keywords appear
%\documentclass[aps,prb,showpacs,reprint,groupedaddress]{revtex4-1}
%\documentclass[aps,showpacs,prl,preprint,superscriptaddress]{revtex4-1}
% Originalmente estava assim:
%\documentclass[nofootinbib,aps,reprint,superscriptaddress]{revtex4}
% Eu resolvi deixar assim:
\documentclass[hidelinks,a4paper,nofootinbib,aps,reprint,superscriptaddress,twocolumn]{revtex4}
\linespread{1.4}

%\documentclass[showpacs,aip,apl,twocolumn,groupedaddress]{revtex4-1}
%\documentclass[aps,showpacs,apl,twocolumn,reprint,groupedaddress]{revtex4-1}
%\documentclass[nature,twocolumn,reprint,groupedaddress]{revtex4-1}
% You should use BibTeX and apsrev.bst for references
% Choosing a journal automatically selects the correct APS
% BibTeX style file (bst file), so only uncomment the line
% below if necessary.
%\bibliographystyle{apsrev4-1}
\bibliographystyle{abntex2-num}

\usepackage{graphicx}
%\usepackage{mathscr}
\usepackage{amsmath,amsfonts}
\usepackage{xcolor}
\usepackage{lineno}
\definecolor{Red}{rgb}{0.9,0.0,0.1}
\definecolor{Blue}{rgb}{0.1,0.1,0.9}
%\usepackage{tabulary}
%\usepackage{subfigure}%
\hyphenation{ma-the-ma-tics e-qui-li-bri-um Bourdieu Ginzburg Adorno Lacey Bradbury Latour Mauss Rosenberg}

\usepackage[brazil]{babel}

\usepackage[utf8]{inputenc}
\usepackage[T1]{fontenc}

\usepackage{natbib}
\usepackage[autostyle]{csquotes}  

\usepackage{hyperref}
\hypersetup{
	pdftitle    = {Atividade},
	pdfsubject  = {MTI},
	pdfauthor   = {Caio César Carvalho Ortega},
	pdfcreator  = {Caio César Carvalho Ortega},
	pdfproducer = {Caio César Carvalho Ortega},
	pdfkeywords = {atividade}
}


\makeatletter
\newcommand*{\citenst}[2][]{%
	\begingroup
	\let\NAT@mbox=\mbox
	\let\@cite\NAT@citenum
	\let\NAT@space\NAT@spacechar
	\let\NAT@super@kern\relax
	\renewcommand\NAT@open{[}%
	\renewcommand\NAT@close{]}%
	\cite[#1]{#2}%
	\endgroup
}
\makeatother

\listfiles

\begin{document}
	%\linenumbers
	% Use the \preprint command to place your local institutional report
	% number in the upper righthand corner of the title page in preprint mode.
	% Multiple \preprint commands are allowed.
	% Use the 'preprintnumbers' class option to override journal defaults
	% to display numbers if necessary
	%\preprint{}
	
	%Title of paper
	\title{O caso do Cantinho do Céu: breve relatório de um esforço de urbanização}
	
	% repeat the \author .. \affiliation etc. as neededcitacao
	% \email, \thanks, \homepage, \altaffiliation all apply to the current
	% author. Explanatory text should go in the []'s, actual e-mail
	% address or url should go in the {}'s for \email and \homepage.
	% Please use the appropriate macro foreach each type of information
	
	% \affiliation command applies to all authors since the last
	% \affiliation command. The \affiliation command should follow the
	% other information
	% \affiliation can be followed by \email, \homepage, \thanks as well.
	%\author{}
	
	\affiliation{Universidade Federal do ABC, Centro de Engenharia, Modelagem e Ciências Sociais Aplicadas, São Bernardo do Campo-SP, Brasil}
	
	\author{Caio César Carvalho Ortega, RA 21038515; Lucas Calefo, RA nnnnnnnn; Raphael Honorato Ribeiro, RA 11100809}
	
	%\date{\today}
	
	\maketitle
	
	\section{Prólogo}
	
	O propósito do presente trabalho é realizar um relatório acerca da reurbanização da área do Cantinho do Céu, na capital paulista, como parte das atividades que integram a disciplina de Política Habitacional (ESZT011).
	
	\section{Histórico}
	
	Conforme Silva \cite[p.80]{Silva2016}:
	
	\begin{quote}
		``A ocupação da área do cantinho do Céu, assim como a do Jardim Gaivotas, dá-se em definitivo a partir de 1988, segundo documento da Associação de Moradores do Parque Residencial Cocaia Independente, logo após o surgimento de loteamento Parque dos Lagos, e Lago Azul que aconteceu em 1987 e intensificou-se nos anos 1990, em um momento de grave crise econômica e grande desemprego no país''.
	\end{quote}
	
	No entanto, Silva identificou a partir do contato com um morador local que no final dos anos 1960 a área ainda tinha características rurais \cite[p.80]{Silva2016}. As situações nas décadas seguinte se alteraria \cite[p.82]{Silva2016}:
	
	\begin{quote}
		``Dos anos 1970-1990 a situação mais ao norte da Península do Ribeirão Cocaia ainda era pior, quanto mais ao norte mais se deterioravam as condições de vida da população, justamente onde foram implantados os bairros Cantinho do Céu e Parque Residencial Cocaia.''
	\end{quote}
	
	Silva também salienta que as ocupações são a maioria das moradias no bairro Cantinho do Céu, diferentemente do que acontece no Residencial Cocaia, também integrante da península do Ribeirão Cocaia \cite[p.83]{Silva2016}.
	
	Apesar da presença do transporte metroferroviário, Silva destaca a fragilidade do Cantinho do Céu \cite[p.98]{Silva2016}:
	
	\begin{quote}
		``Muito embora o distrito do Grajaú tenha se desenvolvido muito em algumas de suas regiões, como, por exemplo, nas áreas próximas da linha férrea Estação Grajaú da Companhia Paulista de Transportes Metropolitanos (CPTM), nos bairros Jardim São Paulo, Parque América, ainda existem lugares de grande	precariedade e de espoliação urbano-ambiental, como é o caso de quase todo o bairro Cantinho do Céu e partes do próprio Grajaú (\dots)''.
	\end{quote}

	
	\bibliography{fontes.bib}
	
\end{document}


