\documentclass[a4paper]{article}
\usepackage{amsmath}
\usepackage{tikz}
\usepackage{epigraph}
\usepackage{lipsum}
\usepackage[brazil]{babel}
\usepackage[utf8]{inputenc}
\usepackage[T1]{fontenc}
\usepackage{lmodern}

\renewcommand\epigraphflush{flushright}
\renewcommand\epigraphsize{\normalsize}
\setlength\epigraphwidth{0.7\textwidth}

\definecolor{titlepagecolor}{cmyk}{8,0.5,1,0} % verde
\definecolor{titlepagecolor2}{cmyk}{0,0.3,1,0} % amarelo
\definecolor{titlepagecolor3}{cmyk}{0,.9,.9,.4} % vermelho
\definecolor{titlepagecolor4}{cmyk}{1,.60,0,.40} % azul


% \DeclareFixedFont {<cmd>} {<ENC>} {<family>} {<series>} {<shape>} {<size>} 
% http://tex.loria.fr/ctan-doc/macros/latex/doc/html/fntguide/node11.html
\DeclareFixedFont{\titlefont}{T1}{ppl}{b}{it}{0.5in}
\DeclareFixedFont{\subtitlefont}{T1}{ppl}{b}{it}{0.2in}

\makeatletter                       
\def\printauthor{%                  
	{\large \@author}}              
\makeatother
\author{%
	\textbf{Caio C\'{e}sar Carvalho Ortega} \\
	R.A. 21038515 \\
	\vspace*{0.2cm}
	\textbf{Lucas Calefo} \\
	R.A. 11116216 \\
	\vspace*{0.2cm}
	\textbf{Raphael Honorato Ribeiro} \\
	R.A. 11100809 \\
	%    Author 2 name \\
	%    Department name \\
	%    \texttt{email2@example.com}
}

% The following code is borrowed from: http://tex.stackexchange.com/a/86310/10898

\newcommand\titlepagedecoration{%
	\begin{tikzpicture}[remember picture,overlay,shorten >= -10pt]
	
	\coordinate (aux1) at ([yshift=-15pt]current page.north east);
	\coordinate (aux2) at ([yshift=-410pt]current page.north east);
	\coordinate (aux3) at ([xshift=-4.5cm]current page.north east);
	\coordinate (aux4) at ([yshift=-150pt]current page.north east);
	
	\begin{scope}[titlepagecolor!40,line width=12pt,rounded corners=12pt]
	\draw
	(aux1) -- coordinate (a)
	++(225:5) --
	++(-45:5.1) coordinate (b);
	\draw[shorten <= -10pt]
	(aux3) --
	(a) --
	(aux1);
	\draw[opacity=0.6,titlepagecolor4,shorten <= -10pt]
	(b) --
	++(225:2.2) --
	++(-45:2.2);
	\end{scope}
	\draw[titlepagecolor3,line width=8pt,rounded corners=8pt,shorten <= -10pt]
	(aux4) --
	++(225:0.8) --
	++(-45:0.8);
	\begin{scope}[titlepagecolor2!70,line width=6pt,rounded corners=8pt]
	\draw[shorten <= -10pt]
	(aux2) --
	++(225:3) coordinate[pos=0.45] (c) --
	++(-45:3.1);
	\draw
	(aux2) --
	(c) --
	++(135:2.5) --
	++(45:2.5) --
	++(-45:2.5) coordinate[pos=0.3] (d);   
	\draw 
	(d) -- +(45:1);
	\end{scope}
	\end{tikzpicture}%
}

\begin{document}
	\begin{titlepage}
		
		\noindent
		\titlefont Política\par
		\noindent
		\titlefont Habitacional\par \null
		\noindent
		\subtitlefont Relatório\par
		\epigraph{``Isso nos remete ao agravamento das condições da classe trabalhadora ao
			longo dos anos 1980 e principalmente no fim dessa década e durante os anos 1990.
			Antes os trabalhadores podiam comprar lotes nos loteamentos clandestinos e
			irregulares, após esse período a única opção de moradia para boa parte dos
			trabalhadores passaram a ser as ocupações, pois o salário que ganhavam não lhes
			permitiam mais sequer comprar um lote irregular.''}%
		{\textit{Metrópole corporativa e fragmentada: a urbanização da Península do Ribeirão Cocaia-Grajaú em São Paulo,\\ p. 83}\\ 
			\textsc{Fabiano Leite da Silva}}
		\null\vfill
		\vspace*{1cm}
		\noindent
		\hfill
		\begin{minipage}{0.60\linewidth}
			\begin{flushright}
				\printauthor
			\end{flushright}
		\end{minipage}
		%
		\begin{minipage}{0.02\linewidth}
			\rule{1pt}{125pt}
		\end{minipage}
		\titlepagedecoration
	\end{titlepage}
	%\lipsum[1-2]
\end{document}